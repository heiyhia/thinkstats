% LaTeX source for ``Think Stats:
% Probability and Statistics for Programmers''
% Copyright (c)  2010  Allen B. Downey.

% Creative commons BY-SA
%

%\documentclass[10pt,b5paper]{book}
\documentclass[10pt]{book}
\usepackage[width=5.5in,height=8.5in,
  hmarginratio=3:2,vmarginratio=1:1]{geometry}

% for some of these packages, you might have to install
% texlive-latex-extra (in Ubuntu)

\usepackage{pslatex}
\usepackage{url}
\usepackage{fancyhdr}
\usepackage{graphicx}
\usepackage{amsmath, amsthm, amssymb}
\usepackage{exercise}                        % texlive-latex-extra
\usepackage{makeidx}
\usepackage{setspace}
\usepackage{hevea}                           
\usepackage{upquote}

\title{Think Stats}
\newcommand{\thetitle}{Think Stats: Probability and Statistics for Programmers}
\newcommand{\theversion}{1.0.0}

% these styles get translated in CSS for the HTML version
\newstyle{a:link}{color:black;}
\newstyle{p+p}{margin-top:1em;margin-bottom:1em}
\newstyle{img}{border:0px}

% change the arrows
\setlinkstext
  {\imgsrc[ALT="Previous"]{back.png}}
  {\imgsrc[ALT="Up"]{up.png}}
  {\imgsrc[ALT="Next"]{next.png}}

\makeindex

\begin{document}

\frontmatter

% LATEXONLY

\input{latexonly}

\newtheorem{ex}{Exercise}[chapter]

\begin{latexonly}

\renewcommand{\blankpage}{\thispagestyle{empty} \quad \newpage}

%\blankpage
%\blankpage

% TITLE PAGES FOR LATEX VERSION

%-half title--------------------------------------------------
\thispagestyle{empty}

\begin{flushright}
\vspace*{2.0in}

\begin{spacing}{3}
{\huge Think Stats: Probability and Statistics for Programmers}\\
{\Large }
\end{spacing}

\vspace{0.25in}

Version \theversion

\vfill

\end{flushright}

%--verso------------------------------------------------------

\blankpage
\blankpage
%\clearemptydoublepage
%\pagebreak
%\thispagestyle{empty}
%\vspace*{6in}

%--title page--------------------------------------------------
\pagebreak
\thispagestyle{empty}

\begin{flushright}
\vspace*{2.0in}

\begin{spacing}{3}
{\huge Think Stats}\\
{\Large Probability and Statistics for Programmers}
\end{spacing}

\vspace{0.25in}

Version \theversion

\vspace{1in}


{\Large
Allen Downey\\
}


\vspace{0.5in}

{\Large Green Tea Press}

{\small Needham, Massachusetts}

%\includegraphics[width=1in]{figs/logo1.eps}
\vfill

\end{flushright}


%--copyright--------------------------------------------------
\pagebreak
\thispagestyle{empty}

{\small
Copyright \copyright ~2010 Allen Downey.


\vspace{0.2in}

\begin{flushleft}
Green Tea Press       \\
9 Washburn Ave \\
Needham MA 02492
\end{flushleft}

Replace this.

Permission is granted to copy, distribute, and/or modify this document
under the terms of the GNU Free Documentation License, Version 1.1 or
any later version published by the Free Software Foundation; with no
Invariant Sections, no Front-Cover Texts, and with no Back-Cover Texts.

The GNU Free Documentation License is available from {\tt www.gnu.org}
or by writing to the Free Software Foundation, Inc., 59 Temple Place,
Suite 330, Boston, MA 02111-1307, USA.

The original form of this book is \LaTeX\ source code.  Compiling this
\LaTeX\ source has the effect of generating a device-independent
representation of a textbook, which can be converted to other formats
and printed.

The \LaTeX\ source for this book is available from
\url{http://www.thinkpython.com}

\vspace{0.2in}

} % end small

\end{latexonly}


% HTMLONLY

\begin{htmlonly}

% TITLE PAGE FOR HTML VERSION

{\Large \thetitle}

{\large Allen B. Downey}

Version \theversion

\setcounter{chapter}{-1}

\end{htmlonly}

\chapter{Preface}

\section*{}



Allen B. Downey \\
Needham MA\\

Allen Downey is an Associate Professor of Computer Science at 
the Franklin W. Olin College of Engineering.




\section*{Acknowledgements}



\section*{Contributor List}

\index{contributors}

If you have a suggestion or correction, please send email to 
{\tt feedback@thinkpython.com}.  If I make a change based on your
feedback, I will add you to the contributor list
(unless you ask to be omitted).

If you include at least part of the sentence the
error appears in, that makes it easy for me to search.  Page and
section numbers are fine, too, but not quite as easy to work with.
Thanks!

\small

\begin{itemize}

\item 

% ENDCONTRIB

\end{itemize}

\normalsize

\clearemptydoublepage

% TABLE OF CONTENTS
\begin{latexonly}

\tableofcontents

\clearemptydoublepage

\end{latexonly}

% START THE BOOK
\mainmatter


\chapter{The programmer's path}

This book is about turning data into knowledge.  Data is cheap (at
least relatively); knowledge is harder to come by.

I will present three related pieces:

\begin{description}

\item[Probability] is the study of random events.  Most people have an
  intuitive understanding of degrees of probability, but we will talk
  about how to make quantitative claims about those degrees.

\item[Statistics] is the discipline of using data samples to support
  claims about populations.  Most statistical analysis is based on
  probability, which is why these pieces are usually presented
  together.

\item[Computation] is a tool that is well-suited to quantitative
  analysis, and computers are commonly used to process statistics.
  Also (and more importantly for this book) computational experiments
  are useful for exploring concepts in probability and statistics.

\end{description}

The thesis of this book is that if you know how to program, you can
use that skill to help you understand probability and statistics.
These topics are often presented from a mathematical perspective, and
that approach works well for some people.  But some important ideas
in this area are hard to work with mathematically and relatively
easy to approach computationally.

Both approaches have merits, and the ideal might combine both, but
the goal of this book is to explore the computational path.

The rest of this chapter presents a case study motivated by a question
I heard when my wife and I were expecting our first child: do first
babies tend to arrive late?

\section{Do first babies arrive late?}

If you Google this question, you will find plenty of discussion.
Some people claim it's true, others say it's a myth, and some people
say it's the other way around: first babies come early.

In many of these discussions, people provide data to support their
claims.  I found many examples like these:

\begin{quote}

``My two friends that have given birth recently to their first babies,
BOTH went almost 2 weeks overdue before going into labour or being
induced.''

``My first one came 2 weeks late and now I think the second one is
going to come out two weeks early!!''

``I don't think that can be true because my sister was my mother's
first and she was early, as with many of my cousins.''

\end{quote}

Reports like these are called {\bf anecdotal evidence} because they
are based on data that is unpublished and usually personal.  In casual
conversation, there is nothing wrong with anecdotes, so I don't mean
to pick on the people I quoted.

But we might want evidence that is more persuasive and
an answer that is more reliable.  By those standards, anecdotal
evidence usually fails, because:

\begin{description}

\item[Small number of observations:] If the gestation period for first
  babies is longer for first babies, the difference is probably small
  compared to the natural variation.  In that case, we might have to
  compare a large number of pregnancies to be sure there is really a
  difference (or not).

\item[Selection bias:] People who join a discussion of this question
  might be interested because their first babies were late.  In that
  case the process of selecting data would bias the results.

\item[Confirmation bias:] People who believe the claim might be more
  likely to contribute examples that confirm it.  People who doubt the
  claim are more likely to cite counterexamples.

\item[Inaccuracy:] Anecotes are often personal stories that are
  (deliberately or not) misremembered, misrepresented, repeated
  inaccurately, etc.

\end{description}

So how can we do better?

\section{A statistical approach}

To address the limitations of anecdotes, we will use the tools
of statistics, which include:

\begin{description}

\item[Data collection:]

\item[Exploratory data analysis:]

\item[Descriptive statistics:]

\item[Hypothesis testing:]

\item[Estimation:]

\end{description}


\printindex

\clearemptydoublepage
%\blankpage
%\blankpage
%\blankpage


\end{document}
